% Notes

\rowcolors{1}{notesColor}{notesColor}
\begin{tabularx}{\textwidth}{X}
  \hline

  \noteItem{
    \textbf{Do not under any circumstances update the software} even if prompted!
    The ROS version we use is not compatible with all Ubuntu versions.
  }

  \noteItem{
    The \textbf{abort key for the UI is 'Esc'}.
    This will immediately trigger the abort sequence corresponding to the current phase.
  }

  \noteItem{
    To \textbf{stop a process} in the terminal: CTRL-C
  \\
    To \textbf{save a file}: CTRL-S
  }

  \noteItem{
    Important files that might have to be changed:
    \begin{itemize}
      \item \textbf{/home/dacs/git/configuration-tests/PRO\_DACS-Configuration.xlsx}
            \\
            here you can change controllable actuators per phase and sensor abort thresholds, afterwards do catkin build again

      \item \textbf{/home/dacs/git/user-interface/rosWebPage/ui/configure\_ui.py}
            \\
            here you can change anything related to the UI, in particular which config file is used (this is quite unreadable code so use CTRL-F to find things)
            To apply changes run:
            Cd /home/dacs/git/user-interface/rosWebPage/ui/
            python3 configure\_ui.py

      \item \textbf{/home/dacs/git/software-rpi4/state\_machine/src/:}
            \\
            In this folder you find the state machine list (list of phases) and sequences (watch out for the syntax if you change anything here)
            Check which file is used in state\_machine.cpp
            afterwards do catkin build again

      \item \textbf{/home/dacs/git/software-rpi4/data\_acquisition/src/core/T7\_streaming.py}
            \\
            here you can change the streaming rate in case you get a Labjack buffer error or anything of that kind

      \item \textbf{/home/dacs/git/software-rpi4/data\_acquisition/launch/test.launch}
            \\
            here you can change which ROS modules are started at roslaunch, might be useful to delete the throttling module if it gives any errors for example

      \item \textbf{/home/dacs/git/software-rpi4/throttling\_control/src}
            \\
            In the setpoint.py, control.py and the core folder you can change the setting for the setpoints and the controller parameters.

    \end{itemize}
  }

  \noteItem{
    To see \textbf{error messages regarding the UI} in Google Chrome click on the three dots '…' in the upper right corner, then 'More Tools' and 'Developer Tools'.
    Now you can see a few messages that might help find problems in the Configuration file for example
  }

\end{tabularx}
