% Procedure for preparation

\stepcounter{tableCounter} % Increment counter
\setcounter{rowCounter}{0} % Reset counter
\begin{tabularx}{\textwidth}{|>{\columncolor{tableColumnColor}}c|>{\columncolor{tableColumnColor}}c|>{\columncolor{tableColumnColor}}c|>{\columncolor{tableColumnColor}}c|X|}
  \hline
  \rowcolor{tableHeaderColor}
  ID & CK 1 & CK 2 & CK 3 & Description \\ \hline

  \procedureItem{
    Before starting a test, create a folder for it on sharepoint if it doesn't already exist:
  \\
    Create a folder with the name testtype in /Data/
  \\
    Inside that folder, make a new folder with the test designation.
  }

  \procedureItem{
    Inside the test specific folder, create a folder named Configuration where later the configuration files (\texttt{PRO\_DACS-Configuration}, \texttt{state\_machine\_list}, \texttt{state\_machine\_sequences}) used for that specific test should be saved in
  }

  \procedureItem{
    Create a new folder called Rosbags inside the test specific folder
  }

  \procedureItem{
    Create a new folder called Videos inside the test specific folder
  }

  \procedureItem{
    Create a new folder called Photos inside the test specific folder
  }

  \procedureItem{
    Make sure that the folder \texttt{catkin\_ws/src} contains a folder with the name rosbags.
  \\
    \noindent
  \\
    Before starting a new test, make sure that those folders are empty. If not, go to deinstallation.
  }

  \procedureItem{
    Make sure that the launch file used for the test launches the rosbags node, i.e. you can see a line in the .launch file that looks like this (with the correct path):

    \texttt{
      <node pkg="rosbag" type="record" name="recorder" args="record -a -o /home/\textbf{dacs}/catkin\_ws/src/rosbags/test --split --duration=300"/>
    }
  }
\end{tabularx}
